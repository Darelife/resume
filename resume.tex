\documentclass[10pt, letterpaper]{article}

% Packages:
\usepackage[
    ignoreheadfoot, % set margins without considering header and footer
    top=2 cm, % seperation between body and page edge from the top
    bottom=2 cm, % seperation between body and page edge from the bottom
    left=2 cm, % seperation between body and page edge from the left
    right=2 cm, % seperation between body and page edge from the right
    footskip=1.0 cm, % seperation between body and footer
    % showframe % for debugging 
]{geometry} % for adjusting page geometry
\usepackage{titlesec} % for customizing section titles
\usepackage{tabularx} % for making tables with fixed width columns
\usepackage{array} % tabularx requires this
\usepackage[dvipsnames]{xcolor} % for coloring text
\definecolor{primaryColor}{RGB}{0, 0, 0} % define primary color
\usepackage{enumitem} % for customizing lists
\usepackage{fontawesome5} % for using icons
\usepackage{amsmath} % for math
\usepackage[
    pdftitle={John Doe's CV},
    pdfauthor={John Doe},
    pdfcreator={LaTeX with RenderCV},
    colorlinks=true,
    urlcolor=primaryColor
]{hyperref} % for links, metadata and bookmarks
\usepackage[pscoord]{eso-pic} % for floating text on the page
\usepackage{calc} % for calculating lengths
\usepackage{bookmark} % for bookmarks
\usepackage{lastpage} % for getting the total number of pages
\usepackage{changepage} % for one column entries (adjustwidth environment)
\usepackage{paracol} % for two and three column entries
\usepackage{ifthen} % for conditional statements
\usepackage{needspace} % for avoiding page brake right after the section title
\usepackage{iftex} % check if engine is pdflatex, xetex or luatex

% Ensure that generate pdf is machine readable/ATS parsable:
\ifPDFTeX
    \input{glyphtounicode}
    \pdfgentounicode=1
    \usepackage[T1]{fontenc}
    \usepackage[utf8]{inputenc}
    \usepackage{lmodern}
\fi

\usepackage{charter}

% Some settings:
\raggedright
\AtBeginEnvironment{adjustwidth}{\partopsep0pt} % remove space before adjustwidth environment
\pagestyle{empty} % no header or footer
\setcounter{secnumdepth}{0} % no section numbering
\setlength{\parindent}{0pt} % no indentation
\setlength{\topskip}{0pt} % no top skip
\setlength{\columnsep}{0.15cm} % set column seperation
\pagenumbering{gobble} % no page numbering

\titleformat{\section}{\needspace{4\baselineskip}\bfseries\large}{}{0pt}{}[\vspace{1pt}\titlerule]

\titlespacing{\section}{
    % left space:
    -1pt
}{
    % top space:
    0.3 cm
}{
    % bottom space:
    0.2 cm
} % section title spacing

\renewcommand\labelitemi{$\vcenter{\hbox{\small$\bullet$}}$} % custom bullet points
\newenvironment{highlights}{
    \begin{itemize}[
        topsep=0.10 cm,
        parsep=0.10 cm,
        partopsep=0pt,
        itemsep=0pt,
        leftmargin=0 cm + 10pt
    ]
}{
    \end{itemize}
} % new environment for highlights


\newenvironment{highlightsforbulletentries}{
    \begin{itemize}[
        topsep=0.10 cm,
        parsep=0.10 cm,
        partopsep=0pt,
        itemsep=0pt,
        leftmargin=10pt
    ]
}{
    \end{itemize}
} % new environment for highlights for bullet entries

\newenvironment{onecolentry}{
    \begin{adjustwidth}{
        0 cm + 0.00001 cm
    }{
        0 cm + 0.00001 cm
    }
}{
    \end{adjustwidth}
} % new environment for one column entries

\newenvironment{twocolentry}[2][]{
    \onecolentry
    \def\secondColumn{#2}
    \setcolumnwidth{\fill, 4.5 cm}
    \begin{paracol}{2}
}{
    \switchcolumn \raggedleft \secondColumn
    \end{paracol}
    \endonecolentry
} % new environment for two column entries

\newenvironment{threecolentry}[3][]{
    \onecolentry
    \def\thirdColumn{#3}
    \setcolumnwidth{, \fill, 4.5 cm}
    \begin{paracol}{3}
    {\raggedright #2} \switchcolumn
}{
    \switchcolumn \raggedleft \thirdColumn
    \end{paracol}
    \endonecolentry
} % new environment for three column entries

\newenvironment{header}{
    \setlength{\topsep}{0pt}\par\kern\topsep\centering\linespread{1.5}
}{
    \par\kern\topsep
} % new environment for the header

\newcommand{\placelastupdatedtext}{% \placetextbox{<horizontal pos>}{<vertical pos>}{<stuff>}
  \AddToShipoutPictureFG*{% Add <stuff> to current page foreground
    \put(
        \LenToUnit{\paperwidth-2 cm-0 cm+0.05cm},
        \LenToUnit{\paperheight-1.0 cm}
    ){\vtop{{\null}\makebox[0pt][c]{
        \small\color{gray}\textit{Last updated in September 2024}\hspace{\widthof{Last updated in September 2024}}
    }}}%
  }%
}%

% save the original href command in a new command:
\let\hrefWithoutArrow\href

% new command for external links:


\begin{document}
    \newcommand{\AND}{\unskip
        \cleaders\copy\ANDbox\hskip\wd\ANDbox
        \ignorespaces
    }
    \newsavebox\ANDbox
    \sbox\ANDbox{$|$}

    \begin{header}
        \fontsize{25 pt}{25 pt}\selectfont Prakhar Bhandari

        \vspace{5 pt}

        \normalsize
        \mbox{Jaipur}%
        \kern 5.0 pt%
        \AND%
        \kern 5.0 pt%
        \mbox{\hrefWithoutArrow{mailto:f20230458@goa.bits-pilani.ac.in}{f20230458@goa.bits-pilani.ac.in}}%
        \kern 5.0 pt%
        \AND%
        \kern 5.0 pt%
        \mbox{\hrefWithoutArrow{tel:+91-9136540772}{9136540772}}%
        \kern 5.0 pt%
        \AND%
        % \kern 5.0 pt%
        % \mbox{\hrefWithoutArrow{https://yourwebsite.com/}{yourwebsite.com}}%
        % \kern 5.0 pt%
        % \AND%
        \kern 5.0 pt%
        \mbox{\hrefWithoutArrow{https://www.linkedin.com/in/prakharbhandari13}{linkedin.com/in/prakharbhandari13}}%
        \kern 5.0 pt%
        \AND%
        \kern 5.0 pt%
        \mbox{\hrefWithoutArrow{https://github.com/darelife}{github.com/darelife}}%
    \end{header}

    \vspace{5 pt - 0.3 cm}


    \section{About Me}



        
        \begin{onecolentry}
            I started coding in 2020 (class 10) when the Covid 19 pandemic started. I coded a discord bot with many functionalities, which made me super interested in coding. I initially began with Python. Now, I also know C, C++, JS, React, and Rust. I’m interested in learning more about machine learning models and the inner workings of computers in depth. My hobbies include competitive programming (Codeforces Peak Rating: 1471), playing guitar, and cubing. I grew up in Kuwait, did my class 9 and 10th in Mumbai, and 11th and 12 in Jaipur, while preparing for JEE mains + adv.
        \end{onecolentry}

    \section{Skills}
        \begin{onecolentry}
            Python | C++ | C | Java | Rust
        \end{onecolentry}
        \begin{onecolentry}
            Web Development : MERN Stack
        \end{onecolentry}
        \begin{onecolentry}
            Competitive Programming : Codeforces (Peak 1444 - darelife)
        \end{onecolentry}



    \section{Education}
        
        \begin{itemize}
            \item \textbf{Indian Educational School}, Bharatiya Vidya Bhavan, Kuwait \hfill 2008 -- 2019
            \item \textbf{RN Podar School}, Mumbai \hfill 2019 -- 2021
            \item \textbf{Apex International School}, Jaipur \hfill 2021 -- 2023
            \item \textbf{BITS Pilani KK Birla Goa Campus}, Goa [CSE] \hfill 2023 -- 2027
        \end{itemize}




    
    \section{Projects}



        
        \begin{twocolentry}{
            \href{https://github.com/Darelife/aciax}{Github} : \href{https://aciax.vercel.app/}{Website}
        }
            \textbf{1. Aciax : Organized Display of Educational Resources}\end{twocolentry}

        \vspace{0.10 cm}            
                A website that displays resources and checklists for all subjects, for BITS Pilani, Goa Campus (NextJs, Javascript)
            
        


        \vspace{0.3 cm}

        \begin{twocolentry}{
            \href{https://github.com/Darelife/MultiEquationSolver}{Github}
        }
            \textbf{2. Multi-variable Equation Solver}\end{twocolentry}

        \vspace{0.10 cm}
                Used the Gauss Jordan method to create a reduced row echelon, and solve a multi variable equation. (C++)


        \vspace{0.3 cm}

        \begin{twocolentry}{
            \href{https://github.com/Darelife/synanit2.0}{Github}
        }
            \textbf{3. Synanit : Discord Bot for Codeforces}\end{twocolentry}

        \vspace{0.10 cm}
                A discord.py bot with commands tailored towards Codeforces. (Python)

        \vspace{0.3 cm}

        \begin{twocolentry}{%
            \href{https://github.com/Darelife/HackenzaHackathon}{Data} : 
            \href{https://github.com/arin-r/better-csrankings}{FrontEnd} : 
            \href{https://better-csrankings.vercel.app/}{Website}%
        }
            \textbf{4. Better CS Rankings}
        \end{twocolentry}



        \vspace{0.10 cm}
                Created a lite version of csrankings in a hackathon by only considering Q1 journals, and A/A* conferences as they are highly reputed, while not being too expensive for south asian professors. (Python, NextJs, Typescript)

\end{document}